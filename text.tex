% soctepmlate
% Author: Vojtěch Boček
% Edit by: Jaroslav Páral
% Version: 2018-02-12
% Source code: https://github.com/RoboticsBrno/soctemplate/
% Base on: http://www.jcmm.cz/cz/sablona-soc.html
% License: CC BY 4.0

\documentclass{template/socthesis}

\usepackage{subcaption}
\usepackage{amsmath}
\usepackage{enumitem}

\addbibresource{text.bib}

\titlecz{Modulární stavba soutěžních robotů}
\titleen{Modular construction of competitive robots}
\author{Tomáš Rohlínek}
\field{18} % Obory SOČ: 1 - 18 (http://www.soc.cz/obory-soc/)
\school{Střední průmyslová škola a Vyšší odborná škola Brno, Sokolská, příspěvková organizace}
\mentor{Mgr. Miroslav Burda}
\mentorstatement{Mgr. Miroslava Burdy}

% Změňte, pokud se liší
%\region{Jihomoravský}
\placefooter{Brno 2020}

\begin{document}
	\maketitle
	
	\makecopyrightstatement{V~Drásově}
	
	\makethanks{Děkuji svému vedoucímu Mgr. Miroslavu Burdovi za pomoc, podnětné připomínky a hlavně nekonečnou trpělivost.}
	
	\pagestyle{empty}
	
	\section*{Anotace}
	Cílem této práce je vytvořit sadu univerzálních senzorů pro soutěžní roboty tak, aby jejich instalace, využívání a případná tvorba modulů nových byla co uživatelsky nejpřívětivější.
	
	\subsection*{Klíčová slova}
	robotika; senzory; komunikace; modulární konstrukce
	
	\vspace{20mm}
	
	\section*{Annotation}
	The goal of this work is to create a pack of sensors for competetive robots, while their installation, usage and creation, is as user friendly, as possible. 
	
	\subsection*{Keywords}
	robotics; sensors; communication; modular construction
	
	\newpage
	\pagestyle{plain}
	
	\tableofcontents % vysází obsah
	
	%%% Začátek práce
	\setcounter{figure}{0}
	\setcounter{table}{0}
	\newpage
	
	%%% Úvod
	\chapter*{Úvod}
\addcontentsline{toc}{chapter}{Úvod} % přidá položku úvod do obsahu
%odsazení od vrchu moc velké
V~poslední době se robotické soutěže těší stále většímu zájmu jak veřejnosti, tak konstruktérů.
Senzory dosahují různé kvality a používají různé komunikační protokoly a sběrnice, což může být problémem pro začínající konstruktéry, kteří se díky nejednotné nabídce musí starat o~věci jako duplicitní adresy na sběrnici, kolize dvou knihoven řídících jednu sběrnici a hardwarové problémy dané sběrnice.

Cílem této práce je těmto začínajícím konstruktérům poskytnou nástroj, který většinu problémů se senzorikou vyřeší za ně, pro pokročilejší konstruktéry pak nabízí úsporu času.
Tito konstruktéři nemusí již senzory, které se neustále opakují, stavět, zapojovat a programovat vždy znovu, ale dostanou do rukou téměř plug-n-play řešení.


\newpage
	
	%%% Jak psát
	\chapter{Běžně používané sběrnice}
V oboru amatérské stavby robotů se v praxi používá několik sběrnic a protokolů pro získávání dat.
\begin{itemize}
	\item I$^{2}$C 
	\item UART
		\begin{itemize}
			\item RS-232
			\item RS-485
		\end{itemize}
	\item SPI
	\item 1-Wire
	\item Nesběrnicová komunikace
\end{itemize}

\section{I$^{2}$C}
Snad nejpoužívanější je sběrnice I$^{2}$C,
také známá jako Inter-Intergrated Circuit
\footnote{ Protože je značka I$^{2}$C chráněna, používali ostatní výrobci název TWI, jedná se o prakticky stejnou sběrnici, pouze pod jiným názvem.}.
Tato sběrnice byla vyvinuta firmou Philips, primárně pro připojení periferií, které nevyžadovali vysoké komunikační rychlosti.
Sběrnice podporuje jak multi-master tak multi-slave.
Běžná rychlost je 100kbit/s, ve Fast modu je 400kbit/s. Novější revize pak umožňují až 5 Mbit/s, s touto verzí však nemusí být kompatibilní starší zařízení.
I$^{2}$C používa 7-bitovou adresu, což teoreticky znamená, že je na každou sběrnici možno provozovat až 127 zařízení, prakticky je toto číslo značně nižší.
Maximální délka sběrnice je 1 metr na 100k Baudech, sběrnice však nebyla designovaná na provozu po kabelu, jak ji používá většina amatérských nadšenců.\cite{https://www.nxp.com/docs/en/user-guide/UM10204.pdf}

\begin{table}[]
	\caption{Shrnutí:}
	\centering
	\begin{tabular}{|l|l|l|l|l|} \hline
		Maximální počet zařízení & 127   \\ \hline
		Maximální délka & 1 metr  \\ \hline
		Používanost & 5/5   \\ \hline
		Běžná rychlost & 100kbit/s \\ \hline
		Maximální teoretická rychlost &  5Mbit/s  \\ \hline
		Minimální počet vodičů & 3(SDA, SCL, GND) \\ \hline
	\end{tabular}
\end{table}



\section{UART}
Další hojně používanou sběrnicí je UART, mezi amatérskou komunitou znám též jako sériová linka.
UART ve skutečnosti není sběrnice jako taková, jedná se spíše o něco mezi sběrnicí a protokolem.
Nejběžněji použiváné sběrnice pro UART jsou:
\subsection{RS-232} % (fold)
\label{sub:RS-232}
	RS-232 je implementace UART
% subsection RS-232 (end)
\subsection{RS-485} % (fold)
\label{sub:RS-485}

% subsection RS-485 (end)

% \section{Musíme si dokonale promyslet obsah}
% Jakmile víme, co chceme říci a komu, musíme si rozvrhnout látku.
% Ideální je takové rozvržení, které tvoří logicky přesný a psychologicky stravitelný celek, ve kterém je pro všechno místo a jehož jednotlivé části do sebe přesně zapadají.
% Jsou jasné všechny souvislosti a je zřejmé, co kam patří.

% Abychom tohoto cíle dosáhli, musíme pečlivě organizovat látku.
% Rozhodneme, co budou hlavní kapitoly, co podkapitoly a jaké jsou mezi nimi vztahy.
% Diagramem takové organizace je graf, který je velmi podobný stromu, ale ne řetězci.
% Při organizaci látky je stejně důležitá otázka, co do osnovy zahrnout, jako otázka, co z~ní vypustit.
% Příliš mnoho podrobností může čtenáře právě tak odradit jako žádné detaily.

% \section{Musíme začít psát strukturovaně}
% Máme-li tedy myšlenku, představu o~budoucím čtenáři, cíl a osnovu textu, můžeme začít psát.
% Při psaní prvního konceptu se snažíme zaznamenat všechny své myšlenky a názory vztahující se k~jednotlivým kapitolám a podkapitolám.
% Každou myšlenku musíme vysvětlit, popsat a prokázat.
% Hlavní myšlenku má vždy vyjadřovat hlavní věta a nikoliv věta vedlejší.

	
	%%% Několik formálních pravidel
	\chapter{Běžně používané senzory na soutežních robotech}

Každý robot potřebuje mít způsob interakce s okolím.
Tuto interakci zajišťují právě senzory.
Naprostá většina amatérských týmů nemá prostor ani prostředky vytvářet vlastní senzory.
Používání průmyslových senzorů je znemožňováno několika faktory.
Zřejmě nejzásadnějším je cena, dále pak jejich velikost a hmotnost způsobená jejich robustností a kvalitou provedení.
Týmy jsou tedy nuceny používat hotové destičky, o nichž často nevědí jaké komponenty obsahují a jak fungují.
Senzory se dají rozdělit do několika kategorií:
\begin{itemize}
    \item Senzory vzdálenosti
        \begin{itemize}
            \item Ultrazvukové senzory
            \item IR senzory
            \item Lidar/Radar/Sonar
        \end{itemize}
    \item Senzory barvy
        \begin{itemize}
            \item Black-white senzory
            \item RGB senzory
            \item Kamery
        \end{itemize}
    \item Senzory pohybu
        \begin{itemize}
            \item Akcelerometry
            \item Gyroskopy
            \item Enkodéry
            \item Kompasy
        \end{itemize}
    \item Komplexní polohové senzory
\end{itemize}

\section{Senzory vzdálenosti}

Senzory pro zjišťování vzdálenosti dodávájí robotovi poměrně primitivním způsobem schopnost přibližně určit svou polohu na základě vzdáleností od okolních objektů.
Podmínky pro jejich použití jsou však často velmi specifické a nedají se proto sami o sobě použít pro přesnější lokalizaci.

\subsection{Ultrazvukové senzory}

Asi nejpoužívanějšími senzory vzdálenosti jsou senzory ultrazvukové.
Ty fungují na principu vyslání ultrazvuového pulzu a čekání na jeho návrat.

Nejběžněji používaný z nich je HC-SR04. 
Ten obsahuje ultrazvukový přijímač a vysílač, spolu s dodatečnou elektronikou.
Má čtyři vývody: 
\begin{table}[]
	\caption{Vývody HC-SR04:}
	\centering
	\begin{tabular}{|l|l|l|l|l|} \hline
		GND & společná zem/-   \\ \hline
		VCC/5V & napájení 5V/+  \\ \hline
		Echo & Návrat měřené vzdálenosti   \\ \hline
		Trig & Spouštění meření \\ \hline
	\end{tabular}
\end{table}

Měření započne posláním logické 1 na pin Trig po dobu alespoň 10$\mu$S.
Poté, co Trig opět přepneme na logickou 0, vyšle senzor 8 40-ti kHz pulzů, zárověň nastaví na pinu Echo logickou 1.
Po přijetí odraženého ultrazvukového signálu je na pinu Echo opět nastavena logická 0.
Mikrokontroleru poté pouze zbýva měřit, jak dlouho byla na pinu Echo logická 1, tento čas pak dosadí do rovnice:
\begin{center}
    vzdálenost=(naměřený čas*rychlost zvuku)/2
\end{center} 
Senzor je schopen měřit vzdálenosti od 2cm do 4m, dělá tak na 15$^{\circ}$ úhlu.
Senzor má nevalnou přesnost.
Největší úskalí při používaní tohoto senzoru nastává, pokud je na hřišti více robotů/nesynchronyzovaných senzorů, kdy se senzory mohou navzájem rušit.
Další problém může vyvstat při používaní pouze jednoho mikroprocesoru a několika ultrazvukových senzorů, kdy čas potřebný na změření všech senzorů přesáhne únosnou mez, čímž zásadně prodlouží reakční čas robota.
Tento problém se řeší použitím sekundárního procesoru pro obsloužení měření.
\cite{hc-sr04}

\subsection{IR senzory}
Infračervené senzory fungují vesměs na snejném principu jako ty ultrazvukové, pouze ultrazvukové pulzy jsou nahrazeny infračerveným paprskem.
To s sebou nese oproti ultrazvuku své výhody i nevýhody.
Hlavní výhodou oproti ultrazvuku je menší pravděpodobnost zarušení ambientním signálem, pokud senzor používá modulovaný signál pro měření.
Stejně jako ultrazvukové senzory mohou být i infračervené senzory přehlceny, v tomto případě ale spíše silným ambientním zdrojem, například sluncem, než ostatními senzory.

Nejpoužívanější IR senzor vzdálenosti je FC-51, který má oproti HC-SR04 výrazně větší měřící úhlel 35$^{\circ}$ a výrazně menší rozsah měřitelných vzdáleností, konkrétně 2cm až 30cm.
Na rozdíl od HC-SR04, který měří plynule na celém rozsahu, měří FC-51 pouze přiblížení, podle toho vrací na výstupu 0 a 1, hranice překlopení je nastavitelná pomocí potenciometru nacházejícím se přímo na senzoru.
% \cite{https://artofcircuits.com/product/infrared-obstacle-avoidance-proximity-sensors-module-fc-51}

\subsection{Lidar/Radar/Sonar}
Tato zařízení obsahují běžné senzory vzdálenosti a pouze přidají možnost pohybu těchto senzorů. 
Místo jednosměrného měření pak vytvaří v podstatě mapu prostoru.
Lidar používá k měření laserový, většinou IR, paprsek.
Radar používá rádiové vlny,které mohou částečně pronikat materiály, což umožňuje "vidět" skrz zdi.
Sonar používá k měření zvukové vlny, což je s výhodou využíváno hlavně pod vodou, kde se tyto vlny velmi dobře šíří.
Bohužel zatím neexistuje spolehlivá a levná iterace těchto senzorů, která by se dala požít na soutěžních robotech.

\section{Senzory barvy}
Senzory barvy jsou použitelné pouze v některých soutěžních disciplínách.
Dávají našemu robotovi možnost zjišťovat barvu herních objektů a podložky, což může pomoci jak s orientací, tak s plněním herních úkolů.

\subsection{Black-white senzory}
Tyto senzory využívají světelný zdroj, zpravidla IR nebo bílý, v kombinaci s plynulým světelným senzorem citlivým na vlnovou délku světla zdroje.
Každá látka v závisloti na své barvě pohlcuje a odráží světlo jinak, nám poté zbývá pouze změřit, kolik světla se odrazilo.
Senzory tedy defakto neměří, jestli je povrch černý nebo bílý, ale spíše jestli je světlý či tmavý.
Existuje obrovské množství iterací tohoto senzoru - některé plynulé, jiné digitální s nastavitelnou hranicí překlopení.
B-W senzory se nejčastěji používají v soutěžích jako line follower\footnote{Soutěž, ve které se robot pokouší co nejrychleji projet dráhu vyznačenou nejčastěji černou čárou na podložce. V poslední době se do cesty přidávají také překážky, kterým se robot musí vyhnout.}, kde není potřeba kompletní RGB detekce.

\subsection{RGB senzory}
RGB senzory kombinují více B-W senzorů do jednoho celku.
Existují dvě možnosti, jak toho dosáhnout. 

První možnost používá jeden bílý zdroj světla a více senzorů citlivých na specifické vlnové délky, zpravidla 3 senzory pro RGB, občas se také přidává IR a UV.
Všechny senzory přitom mohou měřit naráz.

Druhá možnost používá jeden senzor se širokým spektrem a více zdrojů světla s různými barvami vyzařovaného světla, opět se nejčastěji používá RGB a případně se k tomu přidává UV a IR.
Tato možnost má o něco pomalejší měření oproti první metodě, neboť zdroje světla se musí ve svícení střídat.

\subsection{Kamery}
Zvláštní případem barevných senzorů jsou kamery.
Ty však vyžadují v závisloti na způsobu použití poměrně velký výpočetní výkon.
To řeší použítí samostatného procesoru pro zpracování obrazu, jako to dělá třeba populární Pixy.\footnote{Nebo můžeme použít aplikaci ve smartphonu, ty pro podobné aplikace mají výpočetního výkonu dostatek. Některé novější by mohli dokonce spustit nějaké Deep learning algoritmy pro zpracování obrazu.}

Kamery poskytují výhodu hlavně co se zorného pole a vzdálenosti od povrchu týká, nemusí totiž být narozdíl od dvou předchozích připevněny do několika milimetrů od měřeného povrchu.

\section{Pohybové senzory} 
Senzory pohybu poskytují robotu schopnost určit, jak se v prostoru pohybuje, některé přímo a některé nepřímo.
Všechny však vyžadují nějaký způsob přepočtu.
Tyto senzory neposkytují vždy přesné údaje, což není způsobeno přímo senzory, ale spíše typem a nedokonalostmi pohybu, který mají měřit.
Polohové senzory, vyjma enkodérů, se z pravidla nepoužívají samostatně, ale v celcích.
Například běžně používaný senzor mpu-6050 spojuje dohromady akcelerometr a gyroskop, zárověň má možnost připojit přímo na sebe kompas, což umožňuje vytvořit 9-osý systém.


\subsection{Akcelerometry}
Akcelerometry měří lineární zrychlení. Ve větším provedení se jedná o závaží na pružině, která je na druhé straně uchycená k pouzdru. 
Lineární zrychlení zjístíme změřením a přepočtem změny délky pružiny, neboli o kolik se pružina zmáčkla/roztáhla, aby uvedla závaží do stejné rychlosti jako má pouzdro.
V menším provedení se využívá piezoelektrického jevu pro měření.
Senzor může měnit hodnoty v závislosti na teplotě, to je potřeba buď kompenzovat ve výpočtu, nebo zanedbat - v případě nepotřeby superpřesných údajů. 
\cite{accel}

\subsection{Gyroskopy}
Gyroskopy měří úhlovou rychlost, případně úhel náklonu.
Ve velkém provedení se jedná o setrvačník připevněný k pouzdru způsobem, který umožňuje rotaci v měřených osách.
Rotující setrvačník si zachová nezávisle na pohybu pouzdra svůj počáteční náklon.
Díky tomu máme referenční objekt, od kterého můžeme měřit náklon.
V menším provedení se používá několik principů, jeden z nich je Disk Resonator Gyroscope, který využívá pro měření Koriolisovy síly.
Pro měření všech 3 os je potřeba více gyroskopů.
\cite{gyro}
\cite{gyro-2}

\subsection{Enkodéry}
Enkodéry slouží převádění mechanického pohybu kol na elektronický signál.
Enkodéry se dají rozdělit na, ty co se používají přímo na pohonu a na vlečné enkodéry.

Enkodéry umístěny na pohonu zabírají méně místa než ty vlečné. 
Naopak pokud proklouzne kolo, či část převodu, nemá tento senzor jak to zjistit.

Vlečné enkodéry mohou mít problém se změnou směru pohybu.
Nejsou však ovlivněny možným proklouznutím hnacího kola, to proto, že měří pohyb relativně k podložce, na rozdíl od těch umístěných na pohonu, které měří pohyb relativně k robotovi.


\subsection{Kompasy}
Stejně jako běžné kompasy se elektronické kompasy používají k zjištění natočení v magnetickém poli Země.

\section{Komplexní polohové senzory}
Když potřebujeme přesné měření polohy, nestačí nám samostatné senzory, musíme vytvořit větší komplexní celek.
Příklad takového systému je Global Positioning System, nebo jeho alternativy jako GLONASS a Galileo.
Tento systém využívá sadu satelitů, které neustále vysílají svou pozici společně s časem odeslání zprávy.
Pokud přijme přijímač data od alespoň 4 satelitů, může pomocí těchto zpráv triangulovat svoji polohu s vysokou přesností.

Další lokalizační systém používá HTC Vive, sada pro vyrtuální realitu, využívající Steam VR tracking.
Ta je pomocí 2 majáčků schopna s přesností na milimetr určit polohu senzoru v prostoru.
Nástroje pro tvorbu vlastních přijímačů jsou navíc volně dostupné, což z toho systému činí perfektní způsob lokalizace objektů v trojdimenzionálním prostoru.
Jediná problematická věc je cena majáčků, které je potřeba koupit, neboť zatím nebyly uvolněny podklady na základě nichž by bylo možné vytvořit vlastní iteraci majáčků.

%https://partner.steamgames.com/vrlicensing#Tracking
%https://www.vive.com/us/accessory/base-station2/

	
	
	%%% Nikdy to nebude naprosto dokonalé
	\chapter{Volba parametrů}
\section{Sběrnice}
Pro svou práci jsem zvolil sběrnici RS-485.
Hlavní důvody pro toto rozhodnutí jsou:
\begin{itemize}
    \item Nezarušitelnost
    \item Počet zařízení
    \item Počet vodičů
    \item Celková robustnost
\end{itemize}

\section{Protokol}
Na sběrnici RS-485 se nabízela možnost použít protokol Modbus, což je průmyslový standard.
Byla to původně i moje volba, ale od jeho použití mě odradila jeho komplexnost.
Rozhodl jsem se proto vytvořit vlastní protokol pracovně pojmenovaný Janus protocol.

\section{Zpracované senzory}
Senzory plánované pro implementaci do této sady zahrnují:
\begin{itemize}
    \item Všesměrový ultrazvokový senzor
    \item Sběrač pro ultrazvukové senzory HC-SR04
    \item Senzor pro Line follower
    \item RGB senzor
    
\end{itemize}
Kromě senzorů mají být součástí sady i další jednotky:
\begin{itemize}
    \item Battery pack - inteligentní baterie
    \item Co-processing unit - přídavná výpočetní jednotka
    \item Rozbočovač - v případě potřeby možnost rozšířit sběrnici o více zařízení
    \item Buffer - Kombinace co-processing unit a rozbočovače
\end{itemize}
\section{Procesory}

Rozhodl jsem se použít pro všechny senzory i řídící jednotku procesory z řady ESP32.
Některé z důvodů:
\begin{itemize}
    \item Build-in podpora pro RS-485 přímo v UART driveru
    \item Kompletní podpora C++v14 na rozdíl od běžně používaných ATMEGA čipů
    \item Kompatibilita se spoustou rozhraní a sběrnic - může sloužit jako převodník mezi ostatními sběrnicemi a Janus protocol
    \item OTA - možnost nahrávání programu přes internet\footnote{Snaha o co nejjednodušší dodatečné opravy systému, kdy si senzory budou sami schopny dostahovat nejnovější firmware, a na obalu nebudou muset být vyvedené programovací konektory.}
    \item WiFi a Bluetooth - V případě nutnosti bezdrátové komunikace
    \item Kompatibilita s Arduino \footnote{Přestože samotný kód protokolu nevyžaduje Arduino a je postaven čistě na ESP-IDF, je Arduino prostředí pro mnoho začátečníků přijemnější pro práci než většina ostatních. Jeho jednoduchost a kompatibilita s jinými čipy však může zásadně limitovat jeho funkce, proto není obsluha protokolu napsaná právě v něm.}
\end{itemize}
	
	%%% Typografické a jazykové zásady
	\chapter{Janus protocol}
Na sběrnici RS-485 se nabízela možnost použít protokol Modbus\cite{modbus}, což je průmyslový standard.
Byla to původně i moje volba, ale v~době, kdy jsem s~prací začínal nebyl ještě Modbus na ESP32 oficiálně podporován.
Rozhodl jsem se proto vytvořit vlastní protokol, pracovně pojmenovaný Janus protocol.
Nyní, i přestože je Modbus na ESP32 již podporová, jsem se rozhodl u~Janus protocol zůstat, protože je oproti Modbus jednoduší.
Zároveň se díky tomu nebudu muset starat o~udržení kompatibility s~Modbusem.
Další z~důvodů byla neexistence některých parametrů v~hlavičce Modbusové zprávy, jako například adresa odesílatele. 

Janus protocol je snaha o~vytvoření uživatelsky co nejpřívětivějšího a zároveň robustního protokolu.
Tento protokol je postavený z~velké části na Avakar protokolu.
Přidává k~němu adresaci zpráv a CRC16.


\section{Základní specifikace}
\begin{itemize}
    \item Multi-slave
    \item Podpora zpráv různých délek -- při zachování minimální délky/hlavičky
    \item Až 254 zařízení (limitováno RS-485 na 32/128 zařízení -- bez použití rozbočovačů)
    \item Jednoduché přidání dalšího senzoru do již zavedeného systému
    \item Nutnost restartu celého systém při nahrazování senzorů, nebo změně řídící jednotky
\end{itemize}

Janus protocol používá 7-bytovou hlavičku následovanou datatovými byty, těch může být teoreticky až 255.
Viz obrázek\ref{fig: Diagram protokolu}.

Hlavička sestává ze startovacího bytu, ten je statický a má hodnotu vždy 0x80.
Následuje adresa adresáta, zařízení, pro které je správa určena.
Pokud je tato adresa rovna 0xFF, jedná se o broadcastovou zprávu, znamenaje, že ji přijmou všechna zařízení, odpovědět na ni však může pouze jedno.
Na třetí pozici je adresa odesílatele zprávy.
Na pozici čtvrté je číslo příkazu, ten říká zařízení jak má se správou naložit.
Následuje byte určující počet datových bytů.
Na šesté a zedmé pozici se nachází 16-bitové CRC.

Datové byty přenáší parametry, pro příkaz určený čísle příkazu v hlavičce.
To mohou být jak nastavovací hodnoty pro senzory, tak hodnoty vracené ze senzorů do řídící jednotky.

\begin{figure}[h]
    \begin{small}
        \begin{center}
            \includegraphics[width=400px]{img/Protocol_diagram.png}
        \end{center}
        \caption{Diagram protokolu}
        \label{fig: Diagram protokolu}
    \end{small}
\end{figure}


Kompletní implementace protokolu pro ESP32 je dostupná na serveru Github.\cite{protocol}



	
	%%% Slovo Romana
	\chapter{Senzory}
Ke dni odevzdání práce je funkční všesměrový ultrazvukový senzor, a to ve dvou iteracích.
Dále je funkční také sběrač ultrazvukových senzorů.

\section{Všesměrový ultrazvukový senzor}
Obě iterace tohoto senzoru používají 8 senzorů HC-SR04 uspořádaných do osmiúhelníku.
Hlavní rozdíl mezi nimi je v možnosnosti použítí.
Zatímco první iterace je nachystaná pro singleplayerové disciplíny,~druhá iterace vyžaduje buď soupeře na hřišti, nebo možnost umístění majáčků okolo hřiště. 
Na Pražském robotickém dni tomuto odpovídá pouze disciplína Roadside Assistance Advanced.

\subsection{Singleplayerové řešení}
Jedná se o osm senzorů HC-SR04 uspořádaných do osmiúhelníku, všechny senzory jsou svedeny do procesoru tak, aby se každý dal číst a aktivovat samostatně.
Senzor musí být na robotovi uchycen nad úrovní všech ostatních jednotek, potřebuje mít 360$^{\circ}$ výhled.
Druhý požadavek na použití tohoto senzoru je výškový rozdíl měřených objektů a objektů, které se na hřišti nacházejí, ale vzdálenost od nich k robotovi měřit nechceme.
Senzor musí být umístěn tak, aby jeho vrchní plocha byla maximámálně na úrovni vrchní plochy měřených objektů, ale zároveň tak, aby při měření nezabíraly nechtěné objekty, což už je ovšem potřeba vyzkoušet na každém hřišti samostatně.

\subsection{Více majáčkové řešení}
Pracovním názvem také Janus omni-ultra.
Tento senzor, nebo lépe řečeno uspořádání, potřebuje ke svému fungování dvě stanice.
Jedna ze stanic je umístěna na robotovi, nazveme ji přijímačem.
Druhá stanice je umístěna na soupeři, případně na pozici pro majáček na okraji hřiště, nazveme ji vysílačem.
Obě stanice jsou založeny na singleplayerovém řešení.

\subsubsection{Vysílač}
Vysílač sestává ze singleplayerové verze, kruhu synchronizačních IR LED, baterie a řídící desky.
Jako baterie je zde použit dvoučlánek baterií 18650.
Řídící deska sestává ze dvou DPS umístěných nad sebou, propojených šestipinovým konektorem.

Na horní desce se náchází procesor a náležitosti potřebné k jeho fungování.
Jako procesor je zde zvolen ATMEGA328, není zde totiž potřeba komunikace po Janus protocol, není tedu nutné používat dražší ESP32.
Pro toto použí je ATMEGA dostačující.
Úkolem procesoru je poslat na 40kHz modulovaný puls ze synchronizačních IR LED společně s vysláním ultrazvukového signálu ze všech HC-SR04 naráz. 
Není třeba měřit návratový čas, protože na vysílači nás nezajímá.
Procesor má zároveň dvě signalizační LED a umožňuje měření baterií, v případě vybítí signalizuje tuto skutečnost.

Na spodní desce se nachází kruh synchronizačních IR LED, spojených na jeden tranzistor.
Dále se zde nachází stabilizátor AMS1117 5V, ten upravuje napětí z baterií pro potřeby procesoru.

\subsubsection{Přijímač}
Přijímač sestává ze singleplayerové verze, kruhu osmi IR přijímačů TSOP4840 a řídící jednotky.
Všechny HC-SR04 mají znefunkčněný vysílač.
Když procesor přijme z IR přijímačů signál, započne měřit čas a odešle vysílací puls do HC-SR04, poté již jen čeká než se změní hodnota některého z Echo pinů na HC-SR04.
V okamžiku, kdy se tak stane, procesor ukončí měření času a zaznamená si, ze kterého HC-SR04 změna přišla.
Díky tomu je schopen senzor zjistit nejen vzdálenost od vysílače, ale i přibližný směr k němu.
Cokoliv co přijde po první změně nás nezajímá, protože se jedná o náhodný odraz od objektů v prostředí.

\section{Sběrač ultrazvukových senzorů}
Tento senzor používá řídící elektroniku singleplayerového řešení všesměrové senzoru.
Hlavní rozdíl oproti němu je možnost libovolného umístění senzorů HC-SR04 na robota.
Důvodem tvorby tohoto senzoru je časová náročnost měření jednotlivých senzorů, která může razantně ovlivnit reakční dobu robota.
Senzor je schopen po softwarovém zapnutí této funkce odesílat interupt signál při překročení nastavené meze přiblížení pro některý z HC-SR04.
	
	%%% Závěr
	\newpage
\chapter*{Závěr}
\addcontentsline{toc}{section}{Závěr}

Janus protocol a hlavně jednotky, které na něm operují se mají stále kam rozvíjet a to jak kvalitatitivně, tak i kvantitativně.
Do budoucna stále zbývá dodělat pořádnou dokumentaci jak pro celý protokol, tak pro jednotlivé senzory.
Hlavně kvůli tomu, aby si již zkušenější uživatelé mohli přidávat sami další senzory do ekosystému a poskytovali tím tak méně zkušeným uživatelům dostupnou sadu senzorů pro vlastní experimentování.
Nedělám si iluze, že by se tento ekosystém případně jeho části používali mimo obor amatérské stavby robotů, případně automatizace, ale k tomu nebyl tento systém nikdy zamýšlen.

	
	\newpage
	\printbibliography[title=Literatura]
	\addcontentsline{toc}{chapter}{Literatura}
	
	\listoffigures
	\addcontentsline{toc}{chapter}{Seznam obrázků}
	
	\listoftables
	\addcontentsline{toc}{chapter}{Seznam tabulek}
	
	\listoflistedequation
	\addcontentsline{toc}{chapter}{Seznam rovnic}
	
\end{document}
